\documentclass[a4paper,12pt]{article}
\title{Lecture 1\\Discrete Computational Structures - Introduction}
\date{}
\author{Raghesh A}
\setlength{\parskip}{.4\baselineskip plus .2\baselineskip minus .1\baselineskip}

\begin{document}
\maketitle
Field of Computer Science is heavily inclined towards Mathematics. For a
computer scientist learning these foundations become inevitable.
So this course!!!.

What does the term Discrete means? What is a Computational Structure? The
answer to the second question is not difficult. We can call a mathematical
structure which assists us to do some sort of computation as a Computational
Structure. But the term Discrete seems to be mysterious!!!

You will be learning the following structures during this course.

\begin{itemize}
\item Logic
\item Relational Structures
\item Group Theory
\item Recurrence Relations
\end{itemize}

Nobody is motivated to learn a topic unless he/she is convinced with
its applications. For each of the about topic the following table shows one
of its applications in Computer Science. For the time being you just remember
this. There are a lot which we will discuss as and when required.

\end{document}
